\documentclass[aspectratio=169, table]{beamer}
\usepackage[utf8]{inputenc}
\usepackage[T1]{fontenc}
\usepackage{graphicx}
\usepackage{fontspec}
\usepackage{xcolor}
\usepackage{tcolorbox}
\usepackage{listings} % Add the listings package
\usepackage{hyperref} % Add the hyperref package

\lstdefinelanguage{JavaScript}{
    keywords={function, var, let, const, if, else, for, while, return, true, false},
    keywordstyle=\color{blue}\bfseries,
    ndkeywords={class, export, boolean, throw, implements, import, this},
    ndkeywordstyle=\color{orange}\bfseries,
    identifierstyle=\color{black},
    sensitive=false,
    comment=[l]{//},
    morecomment=[s]{/*}{*/},
    commentstyle=\color{gray}\ttfamily,
    stringstyle=\color{green}\ttfamily,
}

\lstset{
    breaklines=true,
    language=JavaScript,
    % ... (other style settings)
}

\lstdefinelanguage{PHP}{
    keywords={class, function, echo, if, else, foreach, for, while, return},
    keywordstyle=\color{blue}\bfseries,
    ndkeywords={public, private, protected, static},
    ndkeywordstyle=\color{purple}\bfseries,
    identifierstyle=\color{black},
    sensitive=false,
    comment=[l]{//},
    morecomment=[s]{/*}{*/},
    commentstyle=\color{gray}\ttfamily,
    stringstyle=\color{green}\ttfamily,
}

\lstset{
    breaklines=true,
    language=PHP,
    % ... (other style settings)
}

\setsansfont[
  ItalicFont=fonts/TitilliumWeb-Italic.ttf,
  BoldFont=fonts/TitilliumWeb-Bold.ttf,
  BoldItalicFont=fonts/TitilliumWeb-BoldItalic.ttf,
]{TitilliumWeb-Regular.ttf}

\subtitle{IF120203-Web Programming}
\title{\Huge {\textbf{07: \\Server-side Language}}}
\date[Serial]{\scriptsize {PRU/SPMI/FR-BM-18/0222}}
\author[Pradita]{\small {\textbf{PRADITA UNIVERSITY}}}

\usetheme{Pradita}

\begin{document}
\begin{frame}
    \titlepage
\end{frame}

\begin{frame}{Goals - Using PHP}
    \vskip-1cm
    \begin{itemize}
        \item To learn about PHP and its role in web development
        \item To understand how PHP, a server-side scripting language, is used to create dynamic and interactive web applications
        \item To explore the features of PHP, such as handling form data, working with databases, and creating session management
        \item To apply PHP code to enhance the functionality and interactivity of web pages, including integrating it with JavaScript and jQuery
    \end{itemize}
\end{frame}


\begin{frame}{Introduction to PHP}
    \vskip-0.5cm
    \begin{itemize}
        \item PHP (Hypertext Preprocessor) is a widely used server-side scripting language designed for web development.
        \item It's embedded within HTML code and executed on the server, generating dynamic content before it's sent to the client's web browser.
        \item PHP is versatile and can be used for various tasks such as handling form data, interacting with databases, and managing user sessions.
        \item It's open-source and has a vast community, offering extensive documentation and a wide range of libraries and frameworks.
        \item PHP code is usually embedded in HTML using special tags, such as `<?php ... ?>`, making it easy to mix server-side logic with frontend content.
    \end{itemize}
\end{frame}


% ... (previous frames)

\begin{frame}[fragile]
    \frametitle{JavaScript converter.js - Using jQuery}
    \begin{itemize}
        \item Modify file converter.js
        \item Function for Reverse Button (Next Pages)
        \item Function for Convert Button (Next Pages)
    \end{itemize}
\end{frame}

\begin{frame}[fragile]
    \frametitle{JQuery Reverse Button}
    \begin{lstlisting}[language=JavaScript]
$(document).ready(function () {
  $("#reverse").click(function () {
    const fromSelect = $("#from");
    const toSelect = $("#to");
    const temp = fromSelect.val();
    fromSelect.val(toSelect.val());
    toSelect.val(temp);
  });
// Next is Convert Button
    \end{lstlisting}
\end{frame}

\begin{frame}[fragile]
    \frametitle{JQuery Convert Button - Part 1}
    \begin{lstlisting}[language=JavaScript]
$("#convert").click(function (e) {
    e.preventDefault(); // Prevent the form from submitting
    const fromSelect = $("#from");
    const toSelect = $("#to");
    const amount = $("#amount").val();
    const fromCurrency = fromSelect.val();
    const toCurrency = toSelect.val();
    \end{lstlisting}
\end{frame}

\begin{frame}[fragile]
    \frametitle{JQuery Convert Button - Part 2}
    \begin{lstlisting}[language=JavaScript]
$.ajax({
      url: "converter.php",
      type: "POST",
      data: {
        amount: amount,
        from: fromCurrency,
        to: toCurrency,
      },
    \end{lstlisting}
\end{frame}

\begin{frame}[fragile]
    \frametitle{JQuery Convert Button - Part 3}
    \begin{lstlisting}[language=JavaScript]
success: function (result) {
        const rate = result.rate;
        const convertedAmount = (rate * amount).toLocaleString(undefined, {
          minimumFractionDigits: 2,
          maximumFractionDigits: 2,
        });
    \end{lstlisting}
\end{frame}

\begin{frame}[fragile]
    \frametitle{JQuery Convert Button - Part 4}
    \begin{lstlisting}[language=JavaScript]
let fromCurrencySymbol;
        if (fromCurrency === "IDR") {
          fromCurrencySymbol = "Rp. ";
        } else if (fromCurrency === "USD") {
          fromCurrencySymbol = "$ ";
        } else if (fromCurrency === "EUR") {
          fromCurrencySymbol = "€ ";
        } else if (fromCurrency === "GBP") {
          fromCurrencySymbol = "£ ";
        }
    \end{lstlisting}
\end{frame}

\begin{frame}[fragile]
    \frametitle{JQuery Convert Button - Part 5}
    \begin{lstlisting}[language=JavaScript]
let toCurrencySymbol;
        if (toCurrency === "IDR") {
          toCurrencySymbol = "Rp. ";
        } else if (toCurrency === "USD") {
          toCurrencySymbol = "$ ";
        } else if (toCurrency === "EUR") {
          toCurrencySymbol = "€ ";
        } else if (toCurrency === "GBP") {
          toCurrencySymbol = "£ ";
        }
    \end{lstlisting}
\end{frame}

\begin{frame}[fragile]
    \frametitle{JQuery Convert Button - Part 6}
    \begin{lstlisting}[language=JavaScript]
$("#from-result").text(fromCurrencySymbol + amount);
        $("#from-currency").text(fromCurrency);
        $("#to-result").text(toCurrencySymbol + convertedAmount);
        $("#to-currency").text(toCurrency);
        $("#exchange-rate").text(`1 ${fromCurrency} = ${rate} ${toCurrency}`);
      },
    \end{lstlisting}
\end{frame}

\begin{frame}[fragile]
    \frametitle{JQuery Convert Button - Part 7}
    \begin{lstlisting}[language=JavaScript]
error: function (error) {
        console.error(error);
      },
    });
  });
});
    \end{lstlisting}
\end{frame}

\begin{frame}{Change HTML to PHP}
    \vskip-1cm
    \begin{itemize}
        \item change converter.html to converter.php
        \item adding PHP script on top of the file
    \end{itemize}
\end{frame}

\begin{frame}[fragile]
    \frametitle{Adding PHP Script on Top - Part 1}
    \begin{lstlisting}[language=PHP]
<?php
function convertRate($fromCurrency, $toCurrency)
{
  $apiKey = "fca_live_iq1WvbaON67X9adHTaJiqEszDhP6jtDz7IouUbWv";
  $url = "https://api.freecurrencyapi.com/v1/latest?apikey=$apiKey&currencies=$toCurrency&base_currency=$fromCurrency";

  $ch = curl_init($url);
    \end{lstlisting}
\end{frame}

\begin{frame}[fragile]
    \frametitle{Adding PHP Script on Top - Part 2}
    \begin{lstlisting}[language=PHP]
  curl_setopt($ch, CURLOPT_RETURNTRANSFER, true);
  $response = curl_exec($ch);
  curl_close($ch);
  $data = json_decode($response, true);
  return $data['data'][$toCurrency];
}
    \end{lstlisting}
\end{frame}

\begin{frame}[fragile]
    \frametitle{Adding PHP Script on Top - Part 3}
    \begin{lstlisting}[language=PHP]
if ($_SERVER['REQUEST_METHOD'] == 'POST') {
  $amount = $_POST['amount'];
  $fromCurrency = $_POST['from'];
  $toCurrency = $_POST['to'];
  $rate = convertRate($fromCurrency, $toCurrency);
    \end{lstlisting}
\end{frame}

\begin{frame}[fragile]
    \frametitle{Adding PHP Script on Top - Part 4}
    \begin{lstlisting}[language=PHP]
$response = array(
    'rate' => $rate
  );

  header('Content-Type: application/json');
  echo json_encode($response);
  exit;
}
?>
    \end{lstlisting}
\end{frame}

\begin{frame4}
    \frametitle{Thank You}
\end{frame4}

\end{document}
